
Diário do último ano e poemas selecionados 
Florbela Espanca


\section{Diário do último ano}

\pagebreak

\subsection{JANEIRO 1930}

%Pg. 33

\textbf{11} — Para mim? Para ti? Para ninguém. Quero atirar
para aqui, negligentemente, sem pretensões de estilo,
sem análises filosóficas, o que os ouvidos dos outros
não recolhem: reflexões, impressões, ideias, maneiras
de ver, de sentir — todo o meu espirito paradoxal,
talvez frívolo, talvez profundo.

Foram-se, há muito, os vinte anos, a época das análises,
das complicadas dissecações interiores. Compreendi por
fim que nada compreendi, que mesmo nada poderia ter
compreendido de mim. Restam-me os outros\ldots talvez
por eles possa chegar às infinitas possibilidades do meu
ser misterioso, intangível, secreto.

Nas horas que se desagregam, que desfio entre os meus
dedos parados, sou a que sabe sempre que horas são,
que dia é, o que faz hoje, amanhã, depois. Não sinto
deslizar o tempo através de mim, sou eu que deslizo
através dele e sinto-me passar com a consciência
nítida dos minutos que passam e dos que se vão
seguir. Como compreender a amargura desta 
amargura? Onde páras tu, ó Imprevisto, que vestes de
cor-de-rosa tantas vidas? Deus malicioso e frívolo que
tão lindos mantos teces sobre os ombros das mulheres
que vivem? Para mim és um fantoche, ora amável
ora rabugento, de que eu conheço todos os fios, de
quem eu sei de cor todas as contorções. «Attendre sans
espérer» poderia ser a minha divisa, a divisa do meu


\pagebreak

%Pg. 35


tédio que ainda se dá ao prazer de fazer frases.
Nao tenho nenhum intuito especial ao escrever estas
linhas, não viso nenhum objectivo, não tenho em vista
nenhum fim. Quando morrer, é possível que alguém,
ao ler estes descosidos monólogos, leia o que sente
sem o saber dizer, que essa coisa tão rara neste mundo
— uma alma — se debruce com um pouco de piedade,
um pouco de compreensão, em silêncio, sobre o que eu
fui ou o que julguei ser. E realize o que eu não pude:
\textit{conhecer-me}.

\textbf{12} — Viver não é parar: é continuamente renascer.
As cinzas não aquecem; as águas estagnadas cheiram
mal. Bela! Bela!, não vale recordar o passado! O que
tu foste, só tu o sabes: uma corajosa rapariga, sempre
sincera para consigo mesma.

E consola-te que esse pouco já é alguma coisa. 
Lembra-te que detestas os truques e os prestidigitadores.
Nao há na tua vida um só acto covarde, pois não?
Então que mais queres num mundo em que toda a
gente o é\ldots mais ou menos? Honesta sem 
preconceitos, amorosa sem luxtria, casta sem formalidades,
recta sem princípios e sempre viva, exaltantemente
viva, miraculosamente viva, a palpitar de seiva
quente como as flores selvagens da tua bárbara
charneca!

\pagebreak 

%Pg. 37


\textbf{13} — Os olhos do meu cão enternecem-me. Em que
rosto humano, num outro mundo, vi eu já estes olhos
de veludo doirado, de cantos ligeiramente macerados,
com este mesmo olhar pueril e grave, entre 
interrogativo e ansioso?

\textbf{14} — A minha modesta \textit{chaise} faz-me lembrar — 
«excusez du peu\ldots» — o Estoril em Julho: azul do mar,
passaros esquisitos todos asas, gerânios vermelhos em
grandes umbelas floridas. Passo nela o melhor do meu
tempo. Acendo um cigarro\ldots e o fumo, dum cinzento"-azulado,
eleva-se, quase a direito, até ao tecto, todo
pintalgado duma bizarra folhagem roxa, e de exóticas
rosas em dois tons de alaranjado, flores de papel 
inventadas por crianças para divertir bonecas. E a minha
\textit{réverie} eleva-se com o fumo, adelgaça-se, espraia-se,
espiritualiza-se. E o meu olhar acaricia, de passagem,
o vulto do meu irmão: o meu amigo morto; demora-se,
encantado, nas flores das minhas jarras, agora: andorinhas
todas brancas, lírios roxos feitos de finos crepes
 \textit{georgette}, camélias vestidas de duras sedas
pálidas. A chuva, lá fora, trauteia baixinho a sua clara
e doce cantiga de Inverno, a sua eterna melodia simples
que embala e apazigua. Sinto-me só. Quantas coisas
lindas e tristes eu diria agora a Alguém que não
existe!

\pagebreak

%Pg. 39

\textbf{15} — Como me lembra hoje o jardim da Faculdade!
A minha recordação veste-o do roxo de todas as suas
violetas, nesta evocação de um passado há tanto 
perdido! Maria Albertina, Tarroso, Regado, Camélier,
Fontes, tantas, tantas sombras! Tantos mortos já! 
Jardim por onde ecoaram tantos gritos, tantos risos, tantas
\textit{blagues}, todo o viço e o frémito das nossas inquietas
mocidades, por onde vogaram, confiantes e exaltados,
todos os sonhos das nossas almas que ainda 
acreditavam na glória, na riqueza, na vida e em maravilhosos
destinos de lenda! Não gostaria de o tornar a ver; já
não é o meu jardim, já nao é o nosso jardim; as 
violetas já não são as mesmas violetas, e aquela árvore
grande que parecia debruçar-se a ouvir-nos, meus 
amigos vivos, meus amigos mortos, já decerto nos não
conheceria\ldots

\textbf{21} — E um encanto agora, quase todos os dias
renovado, o meu passeio pela Boavista. Como as
atvores se enfeitam, espreitando a Primavera! 
Polvilham-se de oiro as mimosas, ao crepúsculo riem, num
riso diabólico, as peónias, vestem as magnólias os
seus vestidos de baile: brancos, rosados, cor de lilás\ldots
saias compridas quase a roçar o chão. Para aquela,
pequenina, toda empertigada, em bicos de pés, no
seu tapete de veludo, é talvez este Inverno o seu
vestido de baile. Tão nova ainda! Uma rapa-


\pagebreak

%Pg. 41

riguinha de quinze anos. E que nome terá aquela 
senhora tão alta, toda de roxo, que me diz sempre adeus,
quando passo, em lindos gestos comovidos? E a outra,
mais adiante, de lenço cor-de-rosa amarrado à cabeça
airosa, como uma alentejana? Eu que tenho esgotado
todas as sensações artísticas, sentimentais, intelectuais,
todas as emoções que a minha poderosa imaginação
de criaturinha fantástica e estranha tem sabido bordar
no tecido incolor da minha vida medíocre, não esgotei
ainda, gracas aos deuses, o arrepio de prazer, o 
estremecimento de entusiasmo, este \textit{élan} quase divino,
para tudo o que é belo, grande e puro: flor a abrir ou
tinta de crepúsculo, raminho de árvore, ou gota de
chuva, cores, linhas, perfumes, asas, todas as belas
coisas que me consolam do resto. Serei eu apenas uma
panteísta?

22 — Faço às vezes o gesto de quem segura um filho
ao colo. Um filho, um filho de carne e osso, não me
interessaria talvez, agora\ldots mas sorrio a este, que é
apenas amor nos meus braços.

23 — Endiabrada Bela! Estranha abelha que dos mais
doces cálices só sabe extrair fel! «Para que quer
esta criatura a inteligência, se não há meio de ser
feliz?», dizia, dantes, meu pai, indignado. O 
ingénuo pai de 60 anos, quando é que tu viste ser-


\pagebreak

%Pg. 43

vir a inteligência para tornar feliz alguém? Quando,
ó ingénuo pai de 60 anos?\ldots Só se pode ser feliz
simplificando, simplificando sempre, arrancando, 
diminuindo, esmagando, reduzindo; e a inteligência cria
em volta de nós um mar imenso de ondas, de espumas,
de destrocos, no meio do qual somos depois o 
náufrago que se revolta, que se debate em vão, que não
quer desaparecer sem estreitar de encontro ao peito
qualquer coisa que anda longe: raio de sol em reflexo
de estrelas. E todos os astros moram lá no alto, ó
ingénuo pai de 60 anos!

\textbf{24} — O Diário de Maria Bashkistseff [N.E.] é qualquer coisa
de profundamente triste, de tragicamente humano. Só
não compreendo naquela grande alma o medo da
morte. O espectro da morte, a ideia da morte, 
apavora-a, espanta-a, indigna-a. É a sua única fraqueza.
«Il faudra donc mourir, misérable.» «Mourir? J’en ai
trés peur\ldots Et je ne veux pas.» «Je veux vivre, moi,
quand méme et malgré tout\ldots» «Mon corps pleure et
crie mais quelque chose qui est au-dessus de moi, se
rejouit de vivre, quand méme\ldots» Mas que imensa
alma! Queria o amor, queria a glória, o poder, a
riqueza, queria a felicidade, queria tudo. E morreu com
pouco mais de vinte anos gritando até ao fim que
não queria morrer. Como não compreendeu ela que o

\pagebreak

%Pg. 45

único remate possível à cúpula do seu maravilhoso
palácio de quimeras, de ambição, de amor, de glória,
poderia apenas ser realizado, por essas linhas serenas,
puríssimas, indecifráveis, que só a morte sabe 
esculpir? Os seus vinte anos não chegaram a compreender
o alto e supremo símbolo das mãos que se cruzam,
vazias dessa maré de sonhos, que a vida, em amargo
fluxo e refluxo, leva e traz constantemente. 
Princesinha exilada, porque não soubeste tu murmurar, 
encolhendo os ombros, o teu doce e sereno \textit{nitchevo}
de eslava?\ldots

\subsection{FEVEREIRO}

3 — Chuva, vento, dores, tristeza \ldots e sempre a 
Florbela, a Florbela, a Florbela!! Gostaria de endoidecer:
Carlos Magno ou Semiramis, perseguidora ou 
perseguida, a chorar ou a rir, \textit{Eu} seria outra, outra, outra!
Nao saberia sequer que os meus sonhos eram sonhos:
o mundo estaria todo povoado de verdades. Os meus
exércitos seriam meus, as minhas pedras preciosas seriam
minhas; cóleras, pavores, lágrimas, gargalhadas, tudo isso
seria realmente meu. E uma gota de água seria um astro,
uma espiguinha de erva, uma seara e um ramo de 
árvore, uma floresta. Ser doido é a única forma de \textit{possuir}

\pagebreak

%Pg. 47

e a maneira de ser alguma coisa de firme neste
mundo.

\textbf{4} — Ó Bela imbecil, \textit{trouxa} como tu dizias, irmão
querido, Trouxa\ldots trouxa de farrapos, miseravelmente
esfarrapados. Dentro, há talvez oiro e pedrarias, o
vestido de Cendrillon, a coroa de rosas de Titânia,
a esmeralda de Nero, a lâmpada de Aladim, a taça
do rei de Thule\ldots Quem sabe se ainda ninguém a
desatou?\ldots

6 — A minha vida! Que \textit{gâchis}! Se eu nem mesmo sei
0 que quero!

16 — Que personagem irritante o deste romance idiota
\textit{La ville du Sourire}! «Je me demande vingt fois,
un soir, si je me coucherai a neuf heures ou si je courrai
au dancing et je balance encore, à onze heures, entre
un pyjama posé sur le lit et un smoking posé sur la
chaise\ldots» E gaba-se este pastel de que as mulheres o
perseguiam!\ldots Um homem sem vontade, sem energia,
sem coragem, nunca pode ser verdadeiramente amado.
Ah, ser homem, e um belo impossível trancar-me um
caminho por onde eu quisesse passar!

\pagebreak

%Pg. 49


\textbf{19} — Que me importa a estima dos outros se eu tenho
a minha? Que me importa a mediocridade do mundo
se \textit{Eu} sou \textit{Eu}? Que importa o desalento da vida se há
a morte? Com tantas riquezas porque sentir-me pobre?
E os meus versos e a minha alma, e os meus sonhos,
e os montes e as rosas e a canção dos sapos nas ervas
húmidas e a minha charneca alentejana e os olivais
vestidos de Gata Borralheira e o assombro dos 
crepúsculos e o murmúrio das noites\ldots então isto não é
nada? Napoleão de saias, que impérios desejas? Que
mundos queres conquistar? Estás, decididamente, 
atacada de delírio de grandezas!\ldots

\textbf{22} — O olhar dum bicho comove-me mais 
profundamente que um olhar humano. Há lá dentro 
uma alma que quer falar e não pode, princesa encantada
por qualquer fada má. Num grande esforço de compreensão,
debruço-me, mergulho os meus olhos nos olhos do
meu cão: tu que queres? E os olhos respondem-me e
eu não entendo\ldots Ah, ter quatro patas e compreender
a súplica humilde, a angustiosa ansiedade daquele
olhar! Afinal\ldots de que tendes vós orgulho, ó gentes?\ldots

\textbf{23} — A vida tem a incoeréncia dum sonho. E quem sabe se real-


\pagebreak

%Pg. 51


mente estaremos a dormir e a sonhar e acabaremos por
despertar um dia? Será a esse despertar que os católicos
chamam Deus?

\textbf{28} — Estou tão magrita! A lâmina vai corroendo a
bainha, a pouco e pouco, mas implacavelmente, com
segurança. Devo ter por alma um diamante ou uma
labareda e sinto nela a beleza inquietante e misteriosa
das obras incompletas ou mutiladas.

\subsection{MARÇO}

\textbf{13} — O Luís tem no íntimo, embora o não confesse,
um grande orgulho por não ser capaz de amar 
doidamente uma mulher. Como é que, sendo ele tão 
inteligente, não compreende esta verdade tão simples: que
aquele que não tem nada para dar é que é pobre? Assim,
nas suas aventuras sentimentais, dá, em troca de pedras
preciosas, dinheiro falso e\ldots como cada um dá o que
tem, elas dão sempre pedras preciosas e ele continua a
dar dinheiro falso. E, quando chegar a morte, terá
ignorado dois dos maiores prazeres da vida: o prazer de
possuir pedras preciosas e o prazer de as dar.


\pagebreak

%Pg. 53

\textbf{16} — Imagino-me, em certos momentos, uma 
princesinha, sobre um terraço, sentada num tapete. Em volta\ldots
tanta coisa! Bichos, flores, bonecos\ldots brinquedos. Às
vezes a princesinha aborrece-se de brincar e fica, horas
e horas, esquecida, a cismar num outro mundo onde
houvesse brinquedos maiores, mais belos e mais
sólidos.

\subsection{ABRIL}

\textbf{20} — Ponho-me, às vezes, a olhar para o espelho e a
examinar-me, feição por feição: os olhos, a boca, o 
modelado da fronte, a curva das pálpebras, a linha da face\ldots
E esta amálgama grosseira e feia, grotesca e miserável,
saberia fazer versos? Ah, não! Existe outra coisa\ldots
mas o quê? Afinal, para que pensar? Viver é não saber
que se vive. Procurar o sentido da vida, sem mesmo
sabet se algum sentido tem, é tarefa de poetas e de
neurasténicos. Só uma visão de conjunto pode 
aproximar-se da verdade. Examinar em detalhe é criar novos
detalhes. Por debaixo da cor está o desenho firme e só
se encontra o que se não procura. Porque me não
esqueço eu de viver\ldots para viver?

\pagebreak

%Pg. 55


\textbf{28} — Não tenho forças, não tenho energia, não tenho
coragem para nada. Sinto-me afundar. Sou o ramo de
salgueiro que se inclina e diz que sim a todos os
ventos.

\subsection{MAIO}

\textbf{2} — \textit{La Monnaie de Singe}, de Delarme-Mardrus,
encantou-me, positivamente; sem ser, de maneira 
nenhuma, uma obra-prima é um livro adorável. À parte a
sua estrutura um pouco frágil, os seus exageros, o seu
tom um pouco forçado de demonstração, é realmente
qualquer coisa de bom. A sua «petite fille toute en or»,
longínqua como um ídolo, é um magnífico pretexto para
magníficas páginas cheias de coração e de graça. «La
jalousie et la haine sont des formes de l’hommagge.
C’est un encens amer, mais le plus précieux des encens,
celui que les médiocres ne connaitront jamais.» Como
é verdade! Este livro tem para mim o valor de me ter
debruçado sobre ele como se me tivesse debruçado sobre
a minha alma de rapariga. Lembro-me de ela ter sido,
dantes, um pouco, a alma corajosa e bravia, terna e
inquieta duma «petite fille toute en or». E, também a
mim, foi sempre em «monnaie de singe» que a esmola
da ternura me foi dada\ldots


\pagebreak

%Pg. 57

\subsection{JULHO}

\textbf{16} — Até hoje, todas as minhas cartas de amor não
são mais que a realização da minha necessidade de fazer
frases. Se o Prince Charmant vier, que lhe direi eu
de novo, de sincero, de verdadeiramente sentido? Tão
pobres somos que as mesmas palavras nos servem para
exprimir a mentira e a verdade!

\subsection{AGOSTO}

\textbf{2} — Está escrito que hei-de ser sempre a mesma eterna
isolada\ldots Porquê?

\subsection{SETEMBRO}

\textbf{1} — A águia, será uma águia a valer ou simplesmente
um milhafre?

\textbf{6} — Tenho pela mentira um horror quase físico. 
Sinto-a à distância e agora\ldots neste mesmo momento\ldots
sinto-a vaguear, asquerosa e suja, em volta da minha
alma que vibra no orgulho de ser pura. Se os outros
me não conhecem, eu \textit{conheço-me}, e tenho orgulho,
um incomensurável orgulho em mim!

\pagebreak

%Pg. 59


\subsection{OUTUBRO}

\textbf{8} — Era simplesmente um milhafre\ldots Guardar-me
intacta, como um cristal transparente, para quê? Mas
não imitemos Jeremias\ldots só na alma é que a lama se
não apaga; aquela com que nos salpicam, sai com água
limpa.

\subsection{NOVEMBRO}

\textbf{15} — Não, não e não!

\textbf{20} — A morte definitiva ou a morte transfiguradora?
Mas que importa o que está para além?
Seja o que for, será melhor que o mundo!
Tudo será melhor do que esta vida!

\textbf{24} — Há uma serenidade consciente da sua força na
linha firme daquele perfil. As mãos têm raça e
nobreza; o sorriso, ironia e bondade; os olhos\ldots
não se examinam: deslumbram. Deve ter vivido dez
vidas numa só vida. Há sonhos mortos, como 
violetas esmagadas, na pele fina e macerada das pál-

\pagebreak

%Pg. 61

pebras. Que rastos deixarão na minha vida aqueles
passos, silenciosos e seguros, que sabem o caminho,
todos os caminhos da terra?

\textbf{29} — «La tendresse humaine ne peut s’exprimer que
par un seul geste: celui d’ouvrir et de refermer les
bras.»

\subsection{DEZEMBRO}

\textbf{2} — E não haver gestos novos nem palavras novas!