\part{LIVRO DE MÁGOAS} 

\openany
\chapter{Este livro\ldots{}  }

\begin{verse}
Este livro é de mágoas. Desgraçados\\
Que no mundo passais, chorai ao lê-lo!\\
Somente a vossa dor de Torturados\\
Pode, talvez, senti-lo\ldots{}  e compreendê-lo.

Este livro é para vós. Abençoados\\
Os que o sentirem, sem ser bom nem belo!\\
Bíblia de tristes\ldots{}  Ó Desventurados,\\
Que a vossa imensa dor se acalme ao vê-lo!

Livro de Mágoas\ldots{} 
Dores\ldots{}  Ansiedades!\\
Livro de Sombras\ldots{}  Névoas e Saudades!\\
Vai pelo mundo\ldots{} (Trouxe-o no meu seio\ldots{})

Irmãos na Dor, os olhos rasos de água,\\
Chorai comigo a minha imensa mágoa,\\
Lendo o meu livro só de mágoas cheio!\ldots{}  
\end{verse}

\chapter{Vaidade}

\begin{verse}
Sonho que sou a Poetisa eleita,\\
Aquela que diz tudo e tudo sabe,\\
Que tem a inspiração pura e perfeita,\\
Que reúne num verso a imensidade!

Sonho que um verso meu tem claridade\\
Para encher todo o mundo! E que deleita\\
Mesmo aqueles que morrem de saudade!\\
Mesmo os de alma profunda e insatisfeita!

Sonho que sou Alguém cá neste mundo\ldots{}  \\
Aquela de saber vasto e profundo,\\
Aos pés de quem a Terra anda curvada!

E quando mais no céu eu vou sonhando,\\
E quando mais no alto ando voando,\\
Acordo do meu sonho\ldots{}  E não sou nada!\ldots{}  
\end{verse}

\chapter{Eu}

\begin{verse}
Eu sou a que no mundo anda perdida,\\
Eu sou a que na vida não tem norte,\\
Sou a irmã do Sonho, e desta sorte\\
Sou a crucificada\ldots{} a dolorida\ldots{}  

Sombra de névoa ténue e esvaecida,\\
E que o destino amargo, triste e forte,\\
Impele brutalmente para a morte!\\
Alma de luto sempre incompreendida!\ldots{}  

Sou aquela que passa e ninguém vê\ldots{}\\
Sou a que chamam triste sem o ser\ldots{}\\
Sou a que chora sem saber porquê\ldots{}  

Sou talvez a visão que Alguém sonhou,\\
Alguém que veio ao mundo pra me ver\\
E que nunca na vida me encontrou! 
\end{verse}

\chapter{Castelã da tristeza}

\begin{verse}
Altiva e couraçada de desdém,\\
Vivo sozinha em meu castelo: a Dor!\\
Passa por ele a luz de todo o amor\ldots{}\\
E nunca em meu castelo entrou alguém!

Castelã da Tristeza, vês?\ldots{} A quem?\ldots{}\\
– E o meu olhar é interrogador –\\
Perscruto, ao longe, as sombras do sol-pôr\ldots{}\\
Chora o silêncio\ldots{}  nada\ldots{}  ninguém vem\ldots{}

Castelã da Tristeza, porque choras\\
Lendo, toda de branco, um livro de horas,\\
À sombra rendilhada dos vitrais?\ldots{}

À noite, debruçada, plas ameias,\\
Porque rezas baixinho?\ldots{} Porque anseias?\ldots{}\\
Que sonho afagam tuas mãos reais?\ldots{}
\end{verse}

\chapter{Tortura}

\begin{verse}
Tirar dentro do peito a Emoção,\\
A lúcida Verdade, o Sentimento!\\
– E ser, depois de vir do coração\\
Um punhado de cinza esparso ao vento!\ldots{}

Sonhar um verso de alto pensamento,\\
E puro como um ritmo de oração!\\
– E ser, depois de vir do coração,\\
O pó, o nada, o sonho dum momento\ldots{}

São assim ocos, rudes, os meus versos:\\
Rimas perdidas, vendavais dispersos,
Com que eu iludo os outros, com que minto!

Quem me dera encontrar o verso puro,\\
O verso altivo e forte, estranho e duro,\\
Que dissesse, a chorar, isto que sinto!!
\end{verse}

\chapter{Lágrimas ocultas}

\begin{verse}
Se me ponho a cismar em outras eras\\
Em que ri e cantei, em que era querida,\\
Parece-me que foi noutras esferas,\\
Parece-me que foi numa outra vida\ldots{}

E a minha triste boca dolorida,\\
Que dantes tinha o rir das primaveras,\\
Esbate as linhas graves e severas\\
E cai num abandono de esquecida!

E fico, pensativa, olhando o vago\ldots{}\\
Toma a brandura plácida dum lago\\
O meu rosto de monja de marfim\ldots{}\\

E as lágrimas que choro, branca e calma,\\
Ninguém as vê brotar dentro da alma!\\
Ninguém as vê cair dentro de mim!
\end{verse}

\chapter{Torre de névoa}

\begin{verse}
Subi ao alto, à minha Torre esguia,\\
Feita de fumo, névoas e luar,\\
E pus-me, comovida, a conversar\\
Com os poetas mortos, todo o dia.

Contei-lhes os meus sonhos, a alegria\\
Dos versos que são meus, do meu sonhar,\\
E todos os poetas, a chorar,\\
Responderam-me então: “Que fantasia,

Criança doida e crente! Nós também\\
Tivemos ilusões, como ninguém,\\
E tudo nos fugiu, tudo morreu!\ldots{}”

Calaram-se os poetas, tristemente\ldots{}\\
E é desde então que eu choro amargamente\\
Na minha Torre esguia junto ao céu!\ldots{} 
\end{verse}

\chapter{A minha dor} 

\hfill{}\textit{À você} 

\begin{verse}
A minha Dor é um convento ideal\\
Cheio de claustros, sombras, arcarias,\\
Aonde a pedra em convulsões sombrias\\
Tem linhas dum requinte escultural.

Os sinos têm dobres de agonias\\
Ao gemer, comovidos, o seu mal\ldots{}\\
E todos têm sons de funeral\\
Ao bater horas, no correr dos dias\ldots{}

A minha Dor é um convento. Há lírios\\
Dum roxo macerado de martírios,\\
Tão belos como nunca os viu alguém! 

Nesse triste convento aonde eu moro,\\
Noites e dias rezo e grito e choro,\\
E ninguém ouve\ldots{} ninguém vê\ldots{} ninguém\ldots{}
\end{verse}

\chapter{Dizeres íntimos}

\begin{verse}
É tão triste morrer na minha idade!\\
E vou ver os meus olhos, penitentes\\
Vestidinhos de roxo, como crentes\\
Do soturno convento da Saudade! 

E logo vou olhar (com que ansiedade!\ldots{})\\
As minhas mãos esguias, languescentes,\\
De brancos dedos, uns bebês doentes\\
Que hão-de morrer em plena mocidade! 

E ser-se novo é ter-se o Paraíso,\\
É ter-se a estrada larga, ao sol, florida,\\
Aonde tudo é luz e graça e riso!

E os meus vinte e três anos\ldots{} (Sou tão nova!)\\
Dizem baixinho a rir: “Que linda a vida!\ldots{}”\\
Responde a minha Dor: “Que linda a cova!” 
\end{verse}

\chapter{As minhas ilusões}

\begin{verse}
Hora sagrada dum entardecer\\
De Outono, à beira-mar, cor de safira,\\
Soa no ar uma invisível lira\ldots{}\\
O sol é um doente a enlanguescer\ldots{}

A vaga estende os braços a suster,\\
Numa dor de revolta cheia de ira,\\
A doirada cabeça que delira\\
Num último suspiro, a estremecer!

O sol morreu\ldots{} e veste luto o mar\ldots{}\\
E eu vejo a urna de oiro, a balouçar,\\
À flor das ondas, num lençol de espuma.

As minhas Ilusões, doce tesoiro,\\
Também as vi levar em urna de oiro,\\
No mar da Vida, assim\ldots{} uma por uma\ldots{} 
\end{verse}

\chapter{Neurastenia}

\begin{verse}
Sinto hoje a alma cheia de tristeza!\\
Um sino dobra em mim Ave-Maria!\\
Lá fora, a chuva, brancas mãos esguias,\\
Faz na vidraça rendas de Veneza\ldots{}

O vento desgrenhado chora e reza\\
Por alma dos que estão nas agonias!\\
E flocos de neve, aves brancas, frias,\\
Batem as asas pela Natureza\ldots{}

Chuva\ldots{} tenho tristeza! Mas porquê?!\\
Vento\ldots{} tenho saudades! Mas de quê?!\\
Ó neve que destino triste o nosso!

Ó chuva! Ó vento! Ó neve! Que tortura!\\
Gritem ao mundo inteiro esta amargura,\\
Digam isto que sinto que eu não posso!!\ldots{} 
\end{verse}

\chapter{Pequenina}

\hfill{}\textit{À Maria Helena Falcão}

\begin{verse}
Risques És pequenina e ris\ldots{} A boca breve\\
É um pequeno idílio cor-de-rosa\ldots{}\\
Haste de lírio frágil e mimosa!\\
Cofre de beijos feito sonho e neve!

Doce quimera que a nossa alma deve\\
Ao Céu que assim te faz tão graciosa!\\
Que nesta vida amarga e tormentosa\\
Te fez nascer como um perfume leve!

O ver o teu olhar faz bem à gente\ldots{}\\
E cheira e sabe, a nossa boca, a flores\\
Quando o teu nome diz, suavemente\ldots{}

Pequenina que a Mãe de Deus sonhou,\\
Que ela afaste de ti aquelas dores\\
Que fizeram de mim isto que sou!
\end{verse}

\chapter{A maior tortura}

\hfill{}\textit{A um grande poeta de Portugal!}

\begin{verse}
Na vida, para mim, não há deleite.\\
Ando a chorar convulsa noite e dia\ldots{}\\
E não tenho uma sombra fugidia\\
Onde poise a cabeça, onde me deite!

E nem flor de lilás tenho que enfeite\\
A minha atroz, imensa nostalgia!\ldots{}\\
A minha pobre Mãe tão branca e fria\\
Deu-me a beber a Mágoa no seu leite!

Poeta, eu sou um cardo desprezado,\\
A urze que se pisa sob os pés.\\
Sou, como tu, um riso desgraçado!

Mas a minha tortura inda é maior:\\
Não ser poeta assim como tu és\\
Para gritar num verso a minha Dor!\ldots{}
\end{verse}


\chapter{A Flor do Sonho}

\begin{verse}
A Flor do Sonho, alvíssima, divina,\\
Miraculosamente abriu em mim,\\
Como se uma magnólia de cetim\\
Fosse florir num muro todo em ruína.

Pende em meu seio a haste branda e fina\\
E não posso entender como é que, enfim,\\
Essa tão rara flor abriu assim!\ldots{}\\
Milagre\ldots{} fantasia\ldots{} ou, talvez, sina\ldots{}

Ó Flor que em mim nasceste sem abrolhos,\\
Que tem que sejam tristes os meus olhos\\
Se eles são tristes pelo amor de ti?!\ldots{}

Desde que em mim nasceste em noite\\
calma, Voou ao longe a asa da minha’alma\\
E nunca, nunca mais eu me entendi\ldots{} 
\end{verse}

\chapter{Noite de saudade}

\begin{verse}
A Noite vem poisando devagar\\
Sobre a Terra, que inunda de amargura\ldots{}\\
E nem sequer a bênção do luar\\
A quis tornar divinamente pura\ldots{}

Ninguém vem atrás dela a acompanhar\\
A sua dor que é cheia de tortura\ldots{}\\
E eu oiço a Noite imensa soluçar!\\
E eu oiço soluçar a Noite escura!

Por que és assim tão escura, assim tão triste?!\\
É que, talvez, ó Noite, em ti existe\\
Uma Saudade igual à que eu contenho!

Saudade que eu sei donde me vem\ldots{}\\
Talvez de ti, ó Noite!\ldots{} Ou de ninguém!\ldots{}\\
Que eu nunca sei quem sou, nem o que tenho!! 
\end{verse}

\chapter{Angústia}

\begin{verse}Tortura do pensar! Triste lamento!\\
Quem nos dera calar a tua voz!\\
Quem nos dera cá dentro, muito a sós,\\
Estrangular a hidra num momento!

E não se quer pensar!\ldots{} e o pensamento\\
Sempre a morder-nos bem, dentro de nós\ldots{}\\
Querer apagar no céu – ó sonho atroz! – \\
O brilho duma estrela, com o vento!\ldots{}

E não se apaga, não\ldots{} nada se apaga!\\
Vem sempre rastejando como a vaga\ldots{}\\
Vem sempre perguntando: “O que te resta?\ldots{}” 

Ah! não ser mais que o vago, o infinito!\\
Ser pedaço de gelo, ser granito,\\
Ser rugido de tigre na floresta! 
\end{verse}

\chapter{Amiga}

\begin{verse}
Deixa-me ser a tua amiga, Amor,\\
A tua amiga só, já que não queres\\
Que pelo teu amor seja a melhor,\\
A mais triste de todas as mulheres.

Que só, de ti, me venha mágoa e dor\\
O que me importa a mim?! O que quiseres\\
É sempre um sonho bom! Seja o que for,\\
Bendito sejas tu por mo dizeres!

Beija-me as mãos, Amor, devagarinho\ldots{}\\
Como se os dois nascêssemos irmãos,\\
Aves cantando, ao sol, no mesmo ninho\ldots{}

Beija-mas bem!\ldots{} Que fantasia louca\\
Guardar assim, fechados, nestas mãos\\
Os beijos que sonhei prà minha boca!\ldots{} 
\end{verse}

\chapter{Desejos vãos}

\begin{verse}
Eu queria ser o Mar de altivo porte\\
Que ri e canta, a vastidão imensa!\\
Eu queria ser a Pedra que não pensa,\\
A pedra do caminho, rude e forte!

Eu queria ser o Sol, a luz imensa,\\
O bem do que é humilde e não tem sorte!\\
Eu queria ser a árvore tosca e densa\\
Que ri do mundo vão e até a morte!

Mas o Mar também chora de tristeza\ldots{}\\
As árvores também, como quem reza,\\
Abrem, aos Céus, os braços, como um crente!

E o Sol altivo e forte, ao fim de um dia,\\
Tem lágrimas de sangue na agonia!\\
E as Pedras\ldots{} essas\ldots{} pisa-as toda a gente!\ldots{} 
\end{verse}

\chapter{Pior velhice}

\begin{verse}
Sou velha e triste. Nunca o alvorecer\\
Dum riso são andou na minha boca!\\
Gritando que me acudam, em voz rouca,\\
Eu, náufraga da Vida, ando a morrer!

A Vida, que ao nascer, enfeita e touca\\
De alvas rosas a fronte da mulher,\\
Na minha fronte mística de louca\\
Martírios só poisou a emurchecer!

E dizem que sou nova\ldots{} A mocidade\\
Estará só, então, na nossa idade,\\
Ou está em nós e em nosso peito mora?!

Tenho a pior velhice, a que é mais triste,\\
Aquela onde nem sequer existe\\
Lembrança de ter sido nova\ldots{} outrora\ldots{} 
\end{verse}

\chapter{A um livro}

\begin{verse}
No silêncio de cinzas do meu Ser\\
Agita-se uma sombra de cipreste,\\
Sombra roubada ao livro que ando a ler,\\
A esse livro de mágoas que me deste.

Estranho livro aquele que escreveste,\\
Artista da saudade e do sofrer!\\
Estranho livro aquele em que puseste\\
Tudo o que eu sinto, sem poder dizer!

Leio-o, e folheio, assim, toda a minh’alma!\\
O livro que me deste é meu, e salma\\
As orações que choro e rio e canto!\ldots{}

Poeta igual a mim, ai que me dera\\
Dizer o que tu dizes!\ldots{} Quem soubera\\
Velar a minha Dor desse teu manto!\ldots{} 
\end{verse}

\chapter{Alma perdida}

\begin{verse}
Toda esta noite o rouxinol chorou,\\
Gemeu, rezou, gritou perdidamente!\\
Alma de rouxinol, alma da gente,\\
Tu és, talvez, alguém que se finou!

Tu és, talvez, um sonho que passou,\\
Que se fundiu na Dor, suavemente\ldots{}\\
Talvez sejas a alma, a alma doente\\
Dalguém que quis amar e nunca amou!

Toda a noite choraste\ldots{} e eu chorei\\
Talvez porque, ao ouvir-te, adivinhei\\
Que ninguém é mais triste do que nós!

Contaste tanta coisa à noite calma,\\
Que eu pensei que tu eras a minh’alma\\
Que chorasse perdida em tua voz!\ldots{} 
\end{verse}

\chapter{De joelhos}

\begin{verse}
“Bendita seja a Mãe que te gerou.”\\
Bendito o leite que te fez crescer\\
Bendito o berço aonde te embalou\\
A tua ama, pra te adormecer!

Bendita essa canção que acalentou\\
Da tua vida o doce alvorecer\ldots{}\\
Bendita seja a Lua, que inundou\\
De luz, a Terra, só para te ver\ldots{}

Benditos sejam todos que te amarem,\\
As que em volta de ti ajoelharem\\
Numa grande paixão fervente e louca!

E se mais que eu, um dia, te quiser\\
Alguém, bendita seja essa Mulher,\\
Bendito seja o beijo dessa boca!!
\end{verse}

\chapter{Languidez}

\begin{verse}
Tardes da minha terra, doce encanto,\\
Tardes duma pureza de açucenas,\\
Tardes de sonho, as tardes de novenas,\\
Tardes de Portugal, as tardes de Anto,

Como eu vos quero e amo! Tanto! Tanto!\\
Horas benditas, leves como penas,\\
Horas de fumo e cinza, horas serenas,\\
Minhas horas de dor em que eu sou santo!

Fecho as pálpebras roxas, quase pretas,\\
Que poisam sobre duas violetas,\\
Asas leves cansadas de voar\ldots{}

E a minha boca tem uns beijos mudos\ldots{}\\
E as minhas mãos, uns pálidos veludos,\\
Traçam gestos de sonho pelo ar\ldots{}
\end{verse}

\chapter{Para quê?!}

\begin{verse}
Tudo é vaidade neste mundo vão\ldots{}\\
Tudo é tristeza, tudo é pó, é nada!\\
E mal desponta em nós a madrugada,\\
Vem logo a noite encher o coração!

Até o amor nos mente, essa canção\\
Que o nosso peito ri à gargalhada,\\
Flor que é nascida e logo desfolhada,\\
Pétalas que se pisam pelo chão!\ldots{}

Beijos de amor! Pra quê?!\ldots{} Tristes vaidades!\\
Sonhos que logo são realidades,\\
Que nos deixam a alma como morta!

Só neles acredita quem é louca!\\
Beijos de amor que vão de boca em boca,\\
Como pobres que vão de porta em porta!\ldots{}
\end{verse}

\chapter{Ao vento}
\begin{verse}
O vento passa a rir, torna a passar,\\
Em gargalhadas ásperas de demente;\\
E esta minh’alma trágica e doente\\
Não sabe se há-de rir, se há-de chorar! 

Vento de voz tristonha, voz plangente,\\
Vento que ris de mim sempre a troçar,\\
Vento que ris do mundo e do amor,\\
A tua voz tortura toda a gente!\ldots{}

Vale-te mais chorar, meu pobre amigo!\\
Desabafa essa dor a sós comigo,\\
E não rias assim !\ldots{} Ó vento, chora!

Que eu bem conheço, amigo, esse fadário\\
Do nosso peito ser como um Calvário,\\
e a gente andar a rir pla vida fora!!\ldots{}
\end{verse}

\chapter{Tédio}

\begin{verse}
Passo pálida e triste. Oiço dizer:\\
“Que branca que ela é! Parece morta!”\\
e eu que vou sonhando, vaga, absorta,\\
não tenho um gesto, ou um olhar sequer\ldots{}

Que diga o mundo e a gente o que quiser!\\
– O que é que isso me faz? O que me importa?\ldots{}\\
O frio que trago dentro gela e corta\\
Tudo que é sonho e graça na mulher!

O que é que me importa?! Essa tristeza\\
É menos dor intensa que frieza,\\
É um tédio profundo de viver!

E é tudo sempre o mesmo, eternamente\ldots{}\\
O mesmo lago plácido, dormente\ldots{}\\
E os dias, sempre os mesmos, a correr\ldots{} 
\end{verse}

\chapter{A minha tragédia}

\begin{verse}
Tenho ódio à luz e raiva à claridade\\
Do sol, alegre, quente, na subida.\\
Parece que a minh’alma é perseguida\\
Por um carrasco cheio de maldade!

Ó minha vã, inútil mocidade,\\
Trazes-me embriagada, entontecida!\ldots{}\\
Duns beijos que me deste noutra vida,\\
Trago em meus lábios roxos, a saudade!\ldots{}

Eu não gosto do sol, eu tenho medo\\
Que me leiam nos olhos o segredo\\
De não amar ninguém, de ser assim!

Gosto da Noite imensa, triste, preta,\\
Como esta estranha e doida borboleta\\
Que eu sinto sempre a voltejar em mim!\ldots{}
\end{verse}

\chapter{Sem remédio}

\begin{verse}
Aqueles que me têm muito amor\\
Não sabem o que sinto e o que sou\ldots{}\\
Não sabem que passou, um dia, a Dor\\
À minha porta e, nesse dia, entrou.

E é desde então que eu sinto este pavor,\\
Este frio que anda em mim, e que gelou\\
O que de bom me deu Nosso Senhor!\\
Se eu nem sei por onde ando e onde vou!!

Sinto os passos da Dor, essa cadência\\
Que é já tortura infinda, que é demência!\\
Que é já vontade doida de gritar!

E é sempre a mesma mágoa, o mesmo tédio,\\
A mesma angústia funda, sem remédio,\\
Andando atrás de mim, sem me largar!
\end{verse}

\chapter{Mais triste}

\begin{verse}
É triste, diz a gente, a vastidão\\
Do mar imenso! E aquela voz fatal\\
Com que ele fala, agita o nosso mal!\\
E a Noite é triste como a Extrema-Unção!

É triste e dilacera o coração\\
Um poente do nosso Portugal!\\
E não vêem que eu sou\ldots{} eu\ldots{} afinal,\\
A coisa mais magoada das que são?!\ldots{}

Poentes de agonia trago-os eu\\
Dentro de mim e tudo quanto é meu\\
É um triste poente de amargura!

E a vastidão do Mar, toda essa água\\
Trago-a dentro de mim num mar de Mágoa!\\
E a noite sou eu própria! A Noite escura!! 
\end{verse}

\chapter{Velhinha}

\begin{verse}
Se os que me viram já cheia de graça\\
Olharem bem de frente em mim,\\
Talvez, cheios de dor, digam assim:\\
“Já ela é velha! Como o tempo passa!\ldots{}”

Não sei rir e cantar por mais que faça!\\
Ó minhas mãos talhadas em marfim,\\
Deixem esse fio de oiro que esvoaça!\\
Deixem correr a vida até o fim! 

Tenho vinte e três anos! Sou velhinha!\\
Tenho cabelos brancos e sou crente\ldots{}\\
Já murmuro orações\ldots{} falo sozinha\ldots{}

E o bando cor-de-rosa dos carinhos\\
Que tu me fazes, olho-os indulgente,\\
Como se fosse um bando de netinhos\ldots{} 
\end{verse}

\chapter{Em busca do amor}

\begin{verse}
O meu Destino disse-me a chorar:\\
“Pela estrada da Vida vai andando,\\
E, aos que vires passar, interrogando\\
Acerca do Amor, que hás-de encontrar.”

Fui pela estrada a rir e a cantar,\\
As contas do meu sonho desfilando\ldots{}\\
E noite e dia, à chuva e ao luar,\\
Fui sempre caminhando e perguntando\ldots{}

Mesmo a um velho eu perguntei: “Velhinho,\\
Viste o Amor acaso em teu caminho?”\\
E o velho estremeceu\ldots{} olhou\ldots{} e riu\ldots{}

Agora pela estrada, já cansados,\\
Voltam todos pra trás desanimados\ldots{}\\
E eu paro a murmurar: “Ninguém o viu!\ldots{}”
\end{verse}

\chapter{Impossível}

\begin{verse}
Disseram-me hoje, assim, ao ver-me triste:\\
“Parece Sexta-Feira de Paixão.\\
Sempre a cismar, cismar de olhos no chão,\\
Sempre a pensar na dor que não existe\ldots{}

O que é que tem?! Tão nova e sempre triste!\\
Faça por estar contente! Pois então?!\ldots{}”\\
Quando se sofre, o que se diz é vão\ldots{}\\
Meu coração, tudo, calado, ouviste\ldots{}

Os meus males ninguém mos adivinha\ldots{}\\
A minha Dor não fala, anda sozinha\ldots{}\\
Dissesse ela o que sente! Ai quem me dera!\ldots{}

Os males de Anto toda a gente os sabe!\\
Os meus\ldots{} ninguém\ldots{} A minha Dor não cabe\\
Nos cem milhões de versos que eu fizera!\ldots{}
\end{verse}