\documentclass[showtrims,		% Mostra as marcas de corte em cruz
			   %trimframe,		% Mostra as marcas de corte em linha, para conferência
			   11pt				% 8pt, 9pt, 10pt, 11pt, 12pt, 14pt, 17pt, 20pt 
			   ]{memoir}
\usepackage[brazilian,
			% english,
			% italian,
			% ngerman,
			% french,
			% russian,
			% polutonikogreek
			]{babel}
\usepackage{anyfontsize}			    % para tamanhos de fontes maiores que \Huge 
\usepackage{relsize}					% para aumentar ou diminuir fonte por pontos. Ex. \smaller[1]
\usepackage{fontspec}					% para rodar fontes do sistema
\usepackage[switch]{lineno} 			% para numerar linhas
\usepackage{lipsum}						% para colocar textos lipsum
\usepackage{alltt}						% para colocar espaços duplos. Ex: verso livre
\usepackage{graphicx}					% para colocar imagens
\usepackage{float}						% para flutuar imagens e tabelas 		
\usepackage{lettrine}					% para capsulares
\usepackage{comment}					% para comentar o código em bloco \begin{comment}...
\usepackage{adforn}						% para adornos & glyphs
\usepackage{xcolor}					 	% para texto colorido
\usepackage{array}
\usepackage[babel]{microtype}			% para ajustes finos na mancha
\usepackage{enumerate,enumitem}			% para tipos diferentes de enumeração/formatação ver `edlab-extra.sty`
\usepackage{url}					% para citar sites \url
\usepackage{marginnote}					% para notas laterais
\usepackage{titlesec}					% para produzir os distanciamentos entre pontos no \dotfill
\usepackage{textpos}

%\usepackage{makeidx} 					% para índice remissivo

\usepackage{edlab-penalties}
\usepackage{edlab-git}
\usepackage{edlab-toc}					% define sumário
\usepackage{edlab-extra}				% define epígrafe, quote
\usepackage[largepost]{edlab-margins}
\usepackage[semcabeco, 				% para remover cabeço, sobe mancha e mantem estilos
			]{edlab-sections}			% define pagestyle (cabeço, rodapé e seções)
\usepackage[%
			% notasemlinha 			
			% notalinhalonga
			chicagofootnotes			% para notas com número e ponto cf. man. de Chicago
			]{edlab-footnotes}


% Medidas (ver: memoir p.11 fig.2.3)
\parindent=3ex			% Tamanho da indentação
\parskip=0pt			% Entre parágrafos
\marginparsep=1em		% Entre mancha e nota lateral 
\marginparwidth=4em		% Tamanho da caixa de texto da nota laterial

% Fontes
\newfontfamily\formular{Formular}
\newfontfamily\formularlight{Formular Light}
\setmainfont[Ligatures=TeX,Numbers=OldStyle]{Minion Pro}


% Estilos
\makeoddhead{baruch}{}{\scshape\MakeLowercase{\scshape\MakeTextLowercase{}}}{}
\makeevenhead{baruch}{}{\scshape\MakeLowercase{}}{}
\makeevenfoot{baruch}{}{\footnotesize\thepage}{}
\makeoddfoot{baruch}{}{\footnotesize\thepage}{}
\pagestyle{baruch}		
\headstyles{baruch}

% Testes


\begin{document}


%\input{MUSSUMIPSUM}  		% Teste da classe
% Este documento tem a ver com as partes do LIVRO. 

\thispagestyle{empty}

%\blankpage

% Tamanhos
% \tiny
% \scriptsize
% \footnotesize
% \small 
% \normalsize
% \large 
% \Large 
% \LARGE 
% \huge
% \Huge

% Posicionamento
% \centering 
% \raggedright
% \raggedleft
% \vfill 
% \hfill 
% \vspace{Xcm}   % Colocar * caso esteja no começo de uma página. Ex: \vspace*{...}
% \hspace{Xcm}

% Estilo de página
% \thispagestyle{<<nosso>>}
% \thispagestyle{empty}
% \thispagestyle{plain}  (só número, sem cabeço)
% https://www.overleaf.com/learn/latex/Headers_and_footers

% Compilador que permite usar fonte de sistema: xelatex, lualatex
% Compilador que não permite usar fonte de sistema: latex, pdflatex

% Definindo fontes
% \setmainfont{Times New Roman}  % Todo o texto
% \newfontfamily\avenir{Avenir}  % Contexto

\begingroup\thispagestyle{empty}\vspace*{.05\textheight} 

              \formular
              \Huge
              \noindent
              \textbf{Cantos dos animais\\primordiais}
              
              \vspace{0.3em}

              \noindent\large\textit{Guyra guahu ha mymba ka'aguy ayvu}
                    
\endgroup
\vfill
\pagebreak       % [Frontistício]
%\newcommand{\linhalayout}[2]{{\tiny\textbf{#1}\quad#2\par}}
\newcommand{\linha}[2]{\ifdef{#2}{\linhalayout{#1}{#2}}{}}

\begingroup\tiny
\parindent=0cm
\thispagestyle{empty}

\textbf{edição brasileira©}\quad			 {Hedra \the\year}\\
\textbf{organização e tradução©}\quad		 {Izaque João}\\
\textbf{co-organização}\quad			 	 {Spensy Pimental e Tatiane Klein}\\
%\textbf{tradução}\quad			 			 {Izaque João}\\
%\textbf{posfácio©}\quad			 		 {Fábio Zuker}\\
%\textbf{ilustração©}\quad			 		 {copyrightilustracao}\medskip

%\textbf{edição consultada}\quad			 {edicaoconsultada}\\
%\textbf{primeira edição}\quad			 	 {Acontecimentos}\\
%\textbf{agradecimentos}\quad			 	 {agradecimentos}\\
%\textbf{indicação}\quad			 		 {indicacao}\medskip

\textbf{coordenação da coleção}\quad		 {Luísa Valentini}\\
\textbf{edição}\quad			 			 {Jorge Sallum}\\
\textbf{coedição}\quad			 			 {Suzana Salama}\\
\textbf{assistência editorial}\quad			 {Paulo Henrique Pompermaier}\\
\textbf{revisão do guarani}\quad			 {Arnulfo Morínigo Caballero}\\
\textbf{revisão do português}\quad			 {Spensy Pimentel, Tatiane Klein e Luísa Valentini}\\
\textbf{capa}\quad			 				 {Lucas Kroëff}\\
%\textbf{iconografia}\quad			 		 {iconografia}\\
%\textbf{imagem da capa}\quad			 	 {imagemcapa}\medskip

\textbf{\textsc{isbn}}\quad			 		 {978-65-89705-30-7}

\hspace{-5pt}\begin{tabular}{ll}
\textbf{conselho editorial} & Adriano Scatolin,  \\
							& Antonio Valverde,  \\
							& Caio Gagliardi,    \\
							& Jorge Sallum,      \\
							& Ricardo Valle,     \\
							& Tales Ab'Saber,    \\
							& Tâmis Parron      
\end{tabular}
 
\bigskip
\textit{Grafia atualizada segundo o Acordo Ortográfico da Língua\\
Portuguesa de 1990, em vigor no Brasil desde 2009.}\\

\vfill
\textit{Direitos reservados em língua\\ 
portuguesa somente para o Brasil}\\

\textsc{editora hedra ltda.}\\
R.~Fradique Coutinho, 1139 (subsolo)\\
05416--011 São Paulo \textsc{sp} Brasil\\
Telefone/Fax +55 11 3097 8304\\\smallskip
editora@hedra.com.br\\
www.hedra.com.br\\

Foi feito o depósito legal.

\endgroup
\pagebreak     % [Créditos]
% Tamanhos
% \tiny
% \scriptsize
% \footnotesize
% \small 
% \normalsize
% \large 
% \Large 
% \LARGE 
% \huge
% \Huge

% Posicionamento
% \centering 
% \raggedright
% \raggedleft
% \vfill 
% \hfill 
% \vspace{Xcm}   % Colocar * caso esteja no começo de uma página. Ex: \vspace*{...}
% \hspace{Xcm}

% Estilo de página
% \thispagestyle{<<nosso>>}
% \thispagestyle{empty}
% \thispagestyle{plain}  (só número, sem cabeço)
% https://www.overleaf.com/learn/latex/Headers_and_footers

% Compilador que permite usar fonte de sistema: xelatex, lualatex
% Compilador que não permite usar fonte de sistema: latex, pdflatex

% Definindo fontes
% \setmainfont{Times New Roman}  % Todo o texto
% \newfontfamily\avenir{Avenir}  % Contexto

\begingroup\thispagestyle{empty}\vspace*{.05\textheight} 

              \formular
              \Huge
              \noindent
              \textbf{Memória}
              
              \vspace{0.3em}

              \noindent\large\textit{Diário do último ano e outras recordações}

              \vspace{3em}
              
              \Large\noindent
              Florbela Espanca
              
              \vspace{3em}
              
              \newfontfamily\minion{Minion Pro}
              %{\selectfont\minion\small\noindent Izaque João (\textit{organização e tradução})}

              \bigskip

              \noindent
              {\selectfont\minion\small\noindent 1ª edição}

              \vfill

              \newfontfamily\timesnewroman{Times New Roman}
              {\noindent\fontsize{30}{40}\selectfont \timesnewroman hedra}

              \noindent{\selectfont\minion\small
              \noindent São Paulo \quad\the\year}

\endgroup
\pagebreak
	       % [folha de rosto]
% nothing			is level -3
% \book				is level -2
% \part				is level -1
% \chapter 			is level 0
% \section 			is level 1
% \subsection 		is level 2
% \subsubsection 	is level 3
% \paragraph 		is level 4
% \subparagraph 	is level 5
\setcounter{secnumdepth}{-2}
\setcounter{tocdepth}{0}

% \renewcommand{\contentsname}{Índex} 	% Trocar nome do sumário para 'Índex'
%\ifodd\thepage\relax\else\blankpage\fi 	% Verifica se página é par e coloca página branca
%\tableofcontents*

\pagebreak
\begingroup \footnotesize \parindent0pt \parskip 5pt \thispagestyle{empty} \vspace*{-0.5\textheight}\mbox{} \vfill
\baselineskip=.92\baselineskip
%\textbf{Florbela Espanca} (1894-1930) 

%\textbf{Memória}  

%\textbf{Izaque João} \lipsum[3]

%\textbf{Coleção Mundo Indígena} reúne materiais produzidos com pensadores de diferentes povos indígenas e pessoas que pesquisam, trabalham ou lutam pela garantia de seus direitos. Os livros foram feitos para serem utilizados pelas comunidades envolvidas na sua produção, e por isso uma parte significativa das obras é bilíngue. Esperamos divulgar a imensa diversidade linguística dos povos indígenas no Brasil, que compreende mais de 150 línguas pertencentes a mais de trinta famílias linguísticas.




\endgroup

\pagebreak\thispagestyle{empty}\movetooddpage
{\begingroup\mbox{}\pagestyle{empty}
\pagestyle{empty} 
% \renewcommand{\contentsname}{Índex} 	% Trocar nome do sumário para 'Índex'
%\ifodd\thepage\relax\else\blankpage\fi 	% Verifica se página é par e coloca página branca
\addtocontents{toc}{\protect\thispagestyle{empty}}
\tableofcontents*\clearpage\endgroup}

\input{APRESENTACAO}
\input{FEITO}
\input{GUARANI}
\input{ATANASIO}
\input{TEXTO}

\pagebreak
\blankAteven

\pagestyle{empty}

\begingroup
\fontsize{7}{8}\selectfont
{\large\textsc{coleção <<bolso>>}}\\
\begin{enumerate}
\setlength\parskip{4.2pt}
\setlength\itemsep{-1.4mm}
\item \textit{Don Juan}, Molière
\item \textit{Contos indianos}, Mallarmé
\item \textit{Triunfos}, Petrarca
\item \textit{O retrato de Dorian Gray}, Wilde
\item \textit{A história trágica do Doutor Fausto}, Marlowe
\item \textit{Os sofrimentos do jovem Werther}, Goethe
\item \textit{Dos novos sistemas na arte}, Maliévitch
\item \textit{Metamorfoses}, Ovídio
\item \textit{Micromegas e outros contos}, Voltaire
\item \textit{O sobrinho de Rameau}, Diderot
\item \textit{Carta sobre a tolerância}, Locke
\item \textit{Discursos ímpios}, Sade
\item \textit{O príncipe}, Maquiavel
\item \textit{Dao De Jing}, Lao Zi
\item \textit{O fim do ciúme e outros contos}, Proust
\item \textit{Pequenos poemas em prosa}, Baudelaire
\item \textit{Fé e saber}, Hegel
\item \textit{Joana d'Arc}, Michelet
\item \textit{Livro dos mandamentos: 248 preceitos positivos}, Maimônides
\item \textit{O indivíduo, a sociedade e o Estado, e outros ensaios}, Emma Goldman
\item \textit{Eu acuso!}, Zola | \textit{O processo do capitão Dreyfus}, Rui Barbosa
\item \textit{Apologia de Galileu}, Campanella 
\item \textit{Sobre verdade e mentira}, Nietzsche
\item \textit{O princípio anarquista e outros ensaios}, Kropotkin
\item \textit{Os sovietes traídos pelos bolcheviques}, Rocker
\item \textit{Poemas}, Byron
\item \textit{Sonetos}, Shakespeare
\item \textit{A vida é sonho}, Calderón
\item \textit{Escritos revolucionários}, Malatesta
\item \textit{Sagas}, Strindberg
\item \textit{O mundo ou tratado da luz}, Descartes
\item \textit{Fábula de Polifemo e Galateia e outros poemas}, Góngora
\item \textit{A vênus das peles}, Sacher{}-Masoch
\item \textit{Escritos sobre arte}, Baudelaire
\item \textit{Cântico dos cânticos}, [Salomão]
\item \textit{Americanismo e fordismo}, Gramsci
\item \textit{O princípio do Estado e outros ensaios}, Bakunin
\item \textit{Balada dos enforcados e outros poemas}, Villon
\item \textit{Sátiras, fábulas, aforismos e profecias}, Da Vinci
\item \textit{O cego e outros contos}, D.H.~Lawrence
\item \textit{Rashômon e outros contos}, Akutagawa
\item \textit{História da anarquia (vol.~1)}, Max Nettlau
\item \textit{Imitação de Cristo}, Tomás de Kempis
\item \textit{O casamento do Céu e do Inferno}, Blake
\item \textit{Flossie, a Vênus de quinze anos}, [Swinburne]
\item \textit{Teleny, ou o reverso da medalha}, [Wilde et al.]
\item \textit{A filosofia na era trágica dos gregos}, Nietzsche
\item \textit{No coração das trevas}, Conrad
\item \textit{Viagem sentimental}, Sterne
\item \textit{Arcana C\oe lestia} e \textit{Apocalipsis revelata}, Swedenborg
\item \textit{Saga dos Volsungos}, Anônimo do séc.~\textsc{xiii}
\item \textit{Um anarquista e outros contos}, Conrad
\item \textit{A monadologia e outros textos}, Leibniz
\item \textit{Cultura estética e liberdade}, Schiller
\item \textit{Poesia basca: das origens à Guerra Civil} 
\item \textit{Poesia catalã: das origens à Guerra Civil} 
\item \textit{Poesia espanhola: das origens à Guerra Civil} 
\item \textit{Poesia galega: das origens à Guerra Civil} 
\item \textit{O pequeno Zacarias, chamado Cinábrio}, E.T.A.~Hoffmann
\item \textit{Entre camponeses}, Malatesta
\item \textit{O Rabi de Bacherach}, Heine
\item \textit{Um gato indiscreto e outros contos}, Saki
\item \textit{Viagem em volta do meu quarto}, Xavier de Maistre 
\item \textit{Hawthorne e seus musgos}, Melville
\item \textit{A metamorfose}, Kafka
\item \textit{Ode ao Vento Oeste e outros poemas}, Shelley
\item \textit{Feitiço de amor e outros contos}, Ludwig Tieck
\item \textit{O corno de si próprio e outros contos}, Sade
\item \textit{Investigação sobre o entendimento humano}, Hume
\item \textit{Sobre os sonhos e outros diálogos}, Borges | Osvaldo Ferrari
\item \textit{Sobre a filosofia e outros diálogos}, Borges | Osvaldo Ferrari
\item \textit{Sobre a amizade e outros diálogos}, Borges | Osvaldo Ferrari
\item \textit{A voz dos botequins e outros poemas}, Verlaine 
\item \textit{Gente de Hemsö}, Strindberg 
\item \textit{Senhorita Júlia e outras peças}, Strindberg 
\item \textit{Correspondência}, Goethe | Schiller
\item \textit{Poemas da cabana montanhesa}, Saigy\=o
\item \textit{Autobiografia de uma pulga}, [Stanislas de Rhodes]
\item \textit{A volta do parafuso}, Henry James
\item \textit{Ode sobre a melancolia e outros poemas}, Keats 
\item \textit{Carmilla --- A vampira de Karnstein}, Sheridan Le Fanu
\item \textit{Pensamento político de Maquiavel}, Fichte
\item \textit{Inferno}, Strindberg
\item \textit{Contos clássicos de vampiro}, Byron, Stoker e outros
\item \textit{O primeiro Hamlet}, Shakespeare
\item \textit{Noites egípcias e outros contos}, Púchkin
\item \textit{Jerusalém}, Blake
\item \textit{As bacantes}, Eurípides
\item \textit{Emília Galotti}, Lessing
\item \textit{Viagem aos Estados Unidos}, Tocqueville
\item \textit{Émile e Sophie ou os solitários}, Rousseau 
\item \textit{Manifesto comunista}, Marx e Engels
\item \textit{A fábrica de robôs}, Karel Tchápek 
\item \textit{Sobre a filosofia e seu método --- Parerga e paralipomena (v.~\textsc{ii}, t.~\textsc{i})}, Schopenhauer 
\item \textit{O novo Epicuro: as delícias do sexo}, Edward Sellon
\item \textit{Revolução e liberdade: cartas de 1845 a 1875}, Bakunin
\item \textit{Sobre a liberdade}, Mill
\item \textit{A velha Izerguil e outros contos}, Górki
\item \textit{Pequeno-burgueses}, Górki
\item \textit{Primeiro livro dos Amores}, Ovídio
\item \textit{Educação e sociologia}, Durkheim
\item \textit{A nostálgica e outros contos}, Papadiamántis 
\item \textit{Lisístrata}, Aristófanes 
\item \textit{A cruzada das crianças/ Vidas imaginárias}, Marcel Schwob
\item \textit{O livro de Monelle}, Marcel Schwob
\item \textit{A última folha e outros contos}, O. Henry
\item \textit{Romanceiro cigano}, Lorca
\item \textit{Sobre o riso e a loucura}, [Hipócrates]
\item \textit{Hino a Afrodite e outros poemas}, Safo de Lesbos 
\item \textit{Anarquia pela educação}, Élisée Reclus 
\item \textit{Ernestine ou o nascimento do amor}, Stendhal
\item \textit{Odisseia}, Homero
\item \textit{O estranho caso do Dr. Jekyll e Mr. Hyde}, Stevenson
\item \textit{História da anarquia (vol.~2)}, Max Nettlau
\item \textit{Sobre a ética --- Parerga e paralipomena (v.~\textsc{ii}, t.~\textsc{ii})}, Schopenhauer 
\item \textit{Contos de amor, de loucura e de morte}, Horacio Quiroga
\item \textit{Memórias do subsolo}, Dostoiévski
\item \textit{A arte da guerra}, Maquiavel
\item \textit{Elogio da loucura}, Erasmo de Rotterdam
\item \textit{Oliver Twist}, Charles Dickens
\item \textit{O ladrão honesto e outros contos}, Dostoiévski
\item \textit{Sobre a utilidade e a desvantagem da histório para a vida}, Nietzsche
\item \textit{Édipo Rei}, Sófocles
\item \textit{Fedro}, Platão
\item \textit{A conjuração de Catilina}, Salústio
\item \textit{O chamado de Cthulhu}, H.\,P.\,Lovecraft
\item \textit{Ludwig Feuerbach e o fim da filosofia clássica alemã}, Engels
\end{enumerate}\medskip

{\large\textsc{coleção <<hedra edições>>}}\\

\begin{enumerate}
\setlength\parskip{4.2pt}
\setlength\itemsep{-1.4mm}
\item \textit{A metamorfose}, Kafka
\item \textit{O príncipe: bilíngue}, Maquiavel
\item \textit{Hino a Afrodite e outros poemas: bilíngue}, Safo de Lesbos 
\item \textit{Jazz rural}, Mário de Andrade
\item \textit{Ludwig Feuerbach e o fim da filosofia clássica alemã}, Friederich Engels
\item \textit{Pr\ae terita}, John Ruskin
\end{enumerate}\medskip

{\large\textsc{coleção <<metabiblioteca>>}}\\

\begin{enumerate}
\setlength\parskip{4.2pt}
\setlength\itemsep{-1.4mm}
\item \textit{O desertor}, Silva Alvarenga
\item \textit{Tratado descritivo do Brasil em 1587}, Gabriel Soares de Sousa
\item \textit{Teatro de êxtase}, Pessoa
\item \textit{Oração aos moços}, Rui Barbosa
\item \textit{A pele do lobo e outras peças}, Artur Azevedo
\item \textit{Tratados da terra e gente do Brasil}, Fernão Cardim 
\item \textit{O Ateneu}, Raul Pompeia
\item \textit{História da província Santa Cruz}, Gandavo
\item \textit{Cartas a favor da escravidão}, Alencar
\item \textit{Pai contra mãe e outros contos}, Machado de Assis
\item \textit{Iracema}, Alencar
\item \textit{Auto da barca do Inferno}, Gil Vicente
\item \textit{Poemas completos de Alberto Caeiro}, Pessoa
\item \textit{A cidade e as serras}, Eça
\item \textit{Mensagem}, Pessoa
\item \textit{Utopia Brasil}, Darcy Ribeiro
\item \textit{Bom Crioulo}, Adolfo Caminha
\item \textit{Índice das coisas mais notáveis}, Vieira
\item \textit{A carteira de meu tio}, Macedo
\item \textit{Elixir do pajé --- poemas de humor, sátira e escatologia}, Bernardo Guimarães
\item \textit{Eu}, Augusto dos Anjos
\item \textit{Farsa de Inês Pereira}, Gil Vicente
\item \textit{O cortiço}, Aluísio Azevedo
\item \textit{O que eu vi, o que nós veremos}, Santos-Dumont
\item \textit{Democracia}, Luiz Gama
\item \textit{Liberdade}, Luiz Gama
\item \textit{A escrava}, Maria Firmina dos Reis
\item \textit{Contos e novelas}, Júlia Lopes de Almeida


\end{enumerate}\medskip

{\large\textsc{<<série largepost>>}}

\begin{enumerate}
\setlength\parskip{4.2pt}
\setlength\itemsep{-1.4mm}
\item \textit{Dao De Jing}, Lao Zi
\item \textit{Escritos sobre literatura}, Sigmund Freud
\item \textit{O destino do erudito}, Fichte
\item \textit{Diários de Adão e Eva}, Mark Twain
\item \textit{Diário de um escritor (1873)}, Dostoiévski
\end{enumerate}


% Série Lovecraft
%\item \textit{A vida de H.P.~Lovecraft}, S.T.~Joshi
%\item \textit{Os melhores contos de H.P.~Lovecraft} 
\medskip

{\large\textsc{<<série sexo>>}}

\begin{enumerate}
\setlength\parskip{4.2pt}
\setlength\itemsep{-1.4mm}

\item \textit{A vênus das peles}, Sacher{}-Masoch
\item \textit{O outro lado da moeda}, Oscar Wilde
\item \textit{Poesia Vaginal}, Glauco Mattoso 
\item \textit{Perversão: a forma erótica do ódio}, Stoller
\item \textit{A vênus de quinze anos}, [Swinburne]
\item \textit{Explosao: romance da etnologia}, Hubert Fichte

\end{enumerate}

\medskip
{\large\textsc{coleção <<que horas são?>>}}

\begin{enumerate}
\setlength\parskip{4.2pt}
\setlength\itemsep{-1.4mm}
\item \textit{Lulismo, carisma pop e cultura anticrítica}, Tales Ab'Sáber
\item \textit{Crédito à morte}, Anselm Jappe
\item \textit{Universidade, cidade e cidadania}, Franklin Leopoldo e Silva
\item \textit{O quarto poder: uma outra história}, Paulo Henrique Amorim
\item \textit{Dilma Rousseff e o ódio político}, Tales Ab'Sáber
\item \textit{Descobrindo o Islã no Brasil}, Karla Lima
\item \textit{Michel Temer e o fascismo comum}, Tales Ab'Sáber
\item \textit{Lugar de negro, lugar de branco?}, Douglas Rodrigues Barros
\item \textit{Machismo, racismo, capitalismo identitário}, Pablo Polese
\item \textit{A linguagem fascista}, Carlos Piovezani \& Emilio Gentile
\end{enumerate}

\medskip
{\large\textsc{coleção <<mundo indígena>>}}

\begin{enumerate}
\setlength\parskip{4.2pt}
\setlength\itemsep{-1.4mm}
\item \textit{A árvore dos cantos}, Pajés Parahiteri
\item \textit{O surgimento dos pássaros}, Pajés Parahiteri
\item \textit{O surgimento da noite}, Pajés Parahiteri
\item \textit{Os comedores de terra}, Pajés Parahiteri
\item \textit{A terra uma só}, Timóteo Verá Tupã Popyguá
\item \textit{Os cantos do homem-sombra}, Mário Pies \& Ponciano Socot
\item \textit{A mulher que virou tatu}, Eliane Camargo
\item \textit{Crônicas de caça e criação}, Uirá Garcia
\item \textit{Círculos de coca e fumaça}, Danilo Paiva Ramos
\item \textit{Nas redes guarani}, Valéria Macedo \& Dominique Tilkin Gallois
\item \textit{Os Aruaques}, Max Schmidt
\item \textit{Cantos dos animais primordiais}, Ava Ñomoandyja Atanásio Teixeira
\end{enumerate}

\medskip
{\large\textsc{coleção <<artecrítica>>}}

\begin{enumerate}
\setlength\parskip{4.2pt}
\setlength\itemsep{-1.4mm}
\item \textit{Dostoiévski e a dialética}, Flávio Ricardo Vassoler
\item \textit{O renascimento do autor}, Caio Gagliardi
\item \textit{O homem sem qualidades à espera de Godot}, Robson de Oliveira
\end{enumerate}

\medskip
{\large\textsc{coleção <<narrativas da escravidão>>}}

\begin{enumerate}
\setlength\parskip{4.2pt}
\setlength\itemsep{-1.4mm}
\item \textit{Incidentes da vida de uma escrava}, Harriet Jacobs
\item \textit{Nascidos na escravidão: depoimentos norte-americanos}, \textsc{wpa}
\item \textit{Narrativa de William W. Brown, escravo fugitivo}, William Wells Brown
\end{enumerate}

\medskip
{\large\textsc{coleção <<walter benjamin>>}}

\begin{enumerate}
\setlength\parskip{4.2pt}
\setlength\itemsep{-1.4mm}
\item \textit{O contador de histórias e outros textos}, Walter Benjamin
\item \textit{Diário parisiense e outros escritos}, Walter Benjamin
\end{enumerate}

\pagebreak	   % [lista de livros publicados]
\pagebreak

\ifodd\thepage\blankpage\fi

\parindent=0pt
\footnotesize\thispagestyle{empty}

% \noindent\textbf{Dados Internacionais de Catalogação na Publicação -- CIP}\\
% \noindent\textbf{(Câmara Brasileira do Livro, SP, Brasil)}\\

% \dotfill\\

% \hspace{20pt}ISBN 978-65-86238-31-0 (Livro do Estudante)

% \hspace{20pt}ISBN 978-65-86238-30-3 (Manual do Professor)\\[6pt]

% \hspace{20pt}\parbox{190pt}{1. Crônicas Brasileira. 2. Contos Brasileiro. 3. Rosa, Alexandre. I. Título.}\\[6pt]

% \hspace{188pt}\textsc{cdd}-B869.8

% \dotfill

% \noindent{}Elaborado por Regina Célia Paiva da Silva CRB -- 1051\\

\mbox{}\vfill
\begin{center}
		\begin{minipage}{.7\textwidth}\tiny\noindent{}
		\centering\tiny
		Adverte-se aos curiosos que se imprimiu este 
		livro na gráfica Meta Brasil, 
		na data de \today, em papel pólen soft, composto em tipologia Minion Pro e Formular, 
		com diversos sofwares livres, 
		dentre eles Lua\LaTeX e git.\\ 
		\ifdef{\RevisionInfo{}}{\par(v.\,\RevisionInfo)}{}\medskip\\\
		\adforn{64}
		\end{minipage}
\end{center}		   % [colofon]


\checkandfixthelayout
\end{document}
